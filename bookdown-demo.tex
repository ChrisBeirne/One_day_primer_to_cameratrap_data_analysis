\documentclass[]{book}
\usepackage{lmodern}
\usepackage{amssymb,amsmath}
\usepackage{ifxetex,ifluatex}
\usepackage{fixltx2e} % provides \textsubscript
\ifnum 0\ifxetex 1\fi\ifluatex 1\fi=0 % if pdftex
  \usepackage[T1]{fontenc}
  \usepackage[utf8]{inputenc}
\else % if luatex or xelatex
  \ifxetex
    \usepackage{mathspec}
  \else
    \usepackage{fontspec}
  \fi
  \defaultfontfeatures{Ligatures=TeX,Scale=MatchLowercase}
\fi
% use upquote if available, for straight quotes in verbatim environments
\IfFileExists{upquote.sty}{\usepackage{upquote}}{}
% use microtype if available
\IfFileExists{microtype.sty}{%
\usepackage[]{microtype}
\UseMicrotypeSet[protrusion]{basicmath} % disable protrusion for tt fonts
}{}
\PassOptionsToPackage{hyphens}{url} % url is loaded by hyperref
\usepackage[unicode=true]{hyperref}
\hypersetup{
            pdftitle={WildCAM: Examples of camera trap data exploration and analysis in R},
            pdfauthor={WildCAM Network},
            pdfborder={0 0 0},
            breaklinks=true}
\urlstyle{same}  % don't use monospace font for urls
\usepackage{natbib}
\bibliographystyle{apalike}
\usepackage{longtable,booktabs}
% Fix footnotes in tables (requires footnote package)
\IfFileExists{footnote.sty}{\usepackage{footnote}\makesavenoteenv{long table}}{}
\usepackage{graphicx,grffile}
\makeatletter
\def\maxwidth{\ifdim\Gin@nat@width>\linewidth\linewidth\else\Gin@nat@width\fi}
\def\maxheight{\ifdim\Gin@nat@height>\textheight\textheight\else\Gin@nat@height\fi}
\makeatother
% Scale images if necessary, so that they will not overflow the page
% margins by default, and it is still possible to overwrite the defaults
% using explicit options in \includegraphics[width, height, ...]{}
\setkeys{Gin}{width=\maxwidth,height=\maxheight,keepaspectratio}
\IfFileExists{parskip.sty}{%
\usepackage{parskip}
}{% else
\setlength{\parindent}{0pt}
\setlength{\parskip}{6pt plus 2pt minus 1pt}
}
\setlength{\emergencystretch}{3em}  % prevent overfull lines
\providecommand{\tightlist}{%
  \setlength{\itemsep}{0pt}\setlength{\parskip}{0pt}}
\setcounter{secnumdepth}{5}
% Redefines (sub)paragraphs to behave more like sections
\ifx\paragraph\undefined\else
\let\oldparagraph\paragraph
\renewcommand{\paragraph}[1]{\oldparagraph{#1}\mbox{}}
\fi
\ifx\subparagraph\undefined\else
\let\oldsubparagraph\subparagraph
\renewcommand{\subparagraph}[1]{\oldsubparagraph{#1}\mbox{}}
\fi

% set default figure placement to htbp
\makeatletter
\def\fps@figure{htbp}
\makeatother

\usepackage{booktabs}
\usepackage{amsthm}
\makeatletter
\def\thm@space@setup{%
  \thm@preskip=8pt plus 2pt minus 4pt
  \thm@postskip=\thm@preskip
}
\makeatother

\title{WildCAM: Examples of camera trap data exploration and analysis in R}
\author{WildCAM Network}
\date{2021-05-11}

\begin{document}
\maketitle

{
\setcounter{tocdepth}{1}
\tableofcontents
}
\chapter{Introduction}\label{introduction}

\section{What this guide is}\label{what-this-guide-is}

\begin{itemize}
\tightlist
\item
  An R code reference manual for basic concepts in the exploration and
  analysis of camera data
\item
  A tool to provide hands on experience of standardised camera trap data
  in R
\item
  A link to resources and papers which showcase the types of analyses
  you can perform (and would maybe want to replicate in the future)
\item
  It will constantly updated and curated by different lab memebers
  through time
\end{itemize}

\section{What this guide isn't}\label{what-this-guide-isnt}

\begin{itemize}
\tightlist
\item
  The ``best'' way to explore / analyse your data (although we hopefully
  provide a good foundation)
\item
  The ``best'' R code for camera traps
\item
  A static document
\end{itemize}

\section{How to use this handbook}\label{how-to-use-this-handbook}

\begin{itemize}
\tightlist
\item
  Browse the pages in the Table of Contents, or use the search box
\item
  You can follow-along with the example data
\item
  Explore addition ``Resources'' at the end of each section for great
  examples
\end{itemize}

\section{Get in touch}\label{get-in-touch}

If you have any questions about this document and the information it
contains, please email use at XXX YYY or submit an issue on our
\href{https://github.com/WildCoLab/WildCAM_Data_Analysis}{data analysis
GitHub page}.

\section{Acknowledgements}\label{acknowledgements}

This guide is produced by the members of the Wildlife Coexistence Lab at
UBC. Check out our \href{}{website}

\chapter{Data standardisation}\label{data-standardisation}

The benefits of `standardizing' our use of camera traps, or other
sensors of biodiversity, are clear: It allows us to rapidly explore and
analyse camera trap data, apply tools created elsewhere, and synthesise
data from different projects. Ultimately resulting in more robust,
repeatable, and timely science and managament decisions.

This is just as true for the process by which we design and implement
our camera trap studies, as it is for how we curate and analyse the
resultant data. That said, it is becoming increasingly clear that we
will \textbf{never} fully standardise camera trap data. Although this is
sounds somewhat defeatist, the number of different data standards which
exist is increasing, and is a reflection of the different requirements
of researchers, practitioners and research aims/objectives of the camera
trapping community.

That doesnt mean we should not standardise our data however! Simply that
we should be clear about the standards we adopt, and stick to them like
glue.

\chapter{Data Exploration}\label{data-exploration}

The most important part of analysing camera trap data is exploring and
checking your data. All to frequently, we find that researchers do not
have a thorough grasp for the quality of their datasets!

\section{Standardised exploration
script}\label{standardised-exploration-script}

In the Wildlife coexistance lab we use a standardised R script to
generate summary information for camera trap projects. The script for
exploring a single site can be found on our
\href{https://github.com/WildCoLab/SingleSiteExploration}{GitHub page}.

\chapter{Methods}\label{methods}

We describe our methods in this chapter.

\chapter{Applications}\label{applications}

Some \emph{significant} applications are demonstrated in this chapter.

\section{Example one}\label{example-one}

\section{Example two}\label{example-two}

\chapter{Final Words}\label{final-words}

We have finished a nice book.

\bibliography{book.bib,packages.bib}

\end{document}
